%XeLaTeX+makeIndex+BibTeX OR LuaLaTeX+...
\documentclass[a4paper,12pt]{article} %14pt - extarticle
\usepackage[utf8]{inputenc} %русский язык, не менять
\usepackage[T2A, T1]{fontenc} %русский язык, не менять
\usepackage[english, russian]{babel} %русский язык, не менять
\usepackage{fontspec} %различные шрифты
\setmainfont{Times New Roman}

%\defaultfontfeatures{Ligatures={TeX},Renderer=Basic}
\usepackage[hyphens]{url} %ссылки \url с переносами
\usepackage{hyperref} %гиперссылки href
\hypersetup{pdfstartview=FitH,  linkcolor=blue, urlcolor=blue, colorlinks=true} %гиперссылки
\usepackage{subfiles}%включение тех-текста
\usepackage{graphicx} %изображения
\usepackage{float}%картинки где угодно
\usepackage{textcomp}


%\usepackage[dvips]{color} %попытка добавить цвета типа OliveGreen. Пропадают все картинки

\usepackage{dsfont}%мат. символы
%\usepackage[style=gost-footnote]{biblatex} %ГОСТ https://www.ctan.org/pkg/biblatex-gost С ним только хуже! Не использовать его!
\newcommand{\ovr}[1]{\overrightarrow{#1}}
\usepackage{listings} %code formatting
\lstset{language=sql,keywordstyle=\color{blue},tabsize=2,breaklines=true,
morekeywords={if, CONCAT, is}}
%хорошая статья об этом пакете: http://mydebianblog.blogspot.com/2012/12/latex.html

\addto\captionsrussian{\def\refname{Использованные источники}}

\begin{document}
\title{Устройство для обучения незрячих чтению рельефным шрифтом <<Тренажёр Брайля>>}
\author{Валерий Зуев}
\maketitle
Добрый день. Я расскажу вам про тренажёр для обучения незрячих чтению по системе Луи Брайля, а в перспективе также и набору символов на компьютере. В качестве вступления послушайте короткую историю о слепых людях. \\
Десять лет сравнительно недалеко отсюда, прямо под окнами моего дома, стоял небольшой завод под названием Контакт. На этом заводе работали слепые. Они делали игрушки, ещё что-то. Слепых селили в домах неподалёку, и для них была специальная инфраструктура - например, говорящие светофоры. Но десять лет назад завод был закрыт, а затем и снесён. На его месте собираются возводить жилой дом.\\
Это  я к тому, насколько тяжело инвалидам по зрению жить нормальной жизнью.\begin{footnotesize}
Конечно, государство не даст слепому умереть с голоду, но взаимодействовать с остальным обществом им крайне нелегко. Из-за вечной бумажной волокиты порой им сложно получить для себя даже то немногое, что положено по закону - а положены им скажем, раз в два года специальные часы с открытыми стрелками, на которых время можно нащупать. Но как слепым постоять за себя во всяких ведомствах? 
\end{footnotesize}\\
%\normalfont
Но хорошим подспорьем в нелёгкой жизни слепого человека стало бы умение читать и писать либо набирать на компьютере специальным точечным шрифтом Брайля. Это не только создаёт психологический комфорт - возможность делать заметки разгружает память и мозг; знание азбуки Брайля даже даёт слепому шанс работать.
\footnotesize
Чтениие книг сейчас у слепых частично заменено прослушиванием аудиокниг, тем не менее шрифт Брайля, ему обучают во всех интернатах для незрячих. 
\normalsize \\
К сожалению, только 30\% незрячих владеют азбукой Брайля. Выходит, меньше трети всех слепых имеют возможность читать и писать. В чём дело? Представьте себя на месте человека, который теряет зрение. 

\end{document}