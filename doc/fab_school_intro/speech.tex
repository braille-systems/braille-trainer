%XeLaTeX+makeIndex+BibTeX OR LuaLaTeX+...
\documentclass[a4paper,12pt]{article} %14pt - extarticle
\usepackage[utf8]{inputenc} %русский язык, не менять
\usepackage[T2A, T1]{fontenc} %русский язык, не менять
\usepackage[english, russian]{babel} %русский язык, не менять
\usepackage{fontspec} %различные шрифты
\setmainfont{Times New Roman}

%\defaultfontfeatures{Ligatures={TeX},Renderer=Basic}
\usepackage[hyphens]{url} %ссылки \url с переносами
\usepackage{hyperref} %гиперссылки href
\hypersetup{pdfstartview=FitH,  linkcolor=blue, urlcolor=blue, colorlinks=true} %гиперссылки
\usepackage{subfiles}%включение тех-текста
\usepackage{graphicx} %изображения
\usepackage{float}%картинки где угодно
\usepackage{textcomp}


%\usepackage[dvips]{color} %попытка добавить цвета типа OliveGreen. Пропадают все картинки

\usepackage{dsfont}%мат. символы
%\usepackage[style=gost-footnote]{biblatex} %ГОСТ https://www.ctan.org/pkg/biblatex-gost С ним только хуже! Не использовать его!
\newcommand{\ovr}[1]{\overrightarrow{#1}}
\usepackage{listings} %code formatting
\lstset{language=sql,keywordstyle=\color{blue},tabsize=2,breaklines=true,
morekeywords={if, CONCAT, is}}
%хорошая статья об этом пакете: http://mydebianblog.blogspot.com/2012/12/latex.html

\addto\captionsrussian{\def\refname{Использованные источники}}

\begin{document}
\title{Устройство для обучения незрячих чтению рельефным шрифтом <<Тренажёр Брайля>>}
\author{Валерий Зуев}
%\maketitle
Добрый день. Я расскажу вам про тренажёр для обучения чтению по системе Луи Брайля незрячих людей. В качестве вступления послушайте короткую историю о слепых людях.\\
Десять лет назад недалеко от Политеха, стоял небольшой завод под названием Контакт. На этом заводе работали слепые. Они делали игрушки, ещё что-то. Слепых селили в домах неподалёку, и для них была специальная инфраструктура. Но лет десять назад завод был закрыт, а затем и снесён. На его месте собираются возводить жилой дом.\\
Это  я к тому, насколько тяжело инвалидам по зрению жить нормальной жизнью. Но хорошим подспорьем для слепого человека стало бы умение читать и писать специальным рельефным шрифтом Брайля. (слайд). Сейчас появились аудиокниги и прочая электроника, но знание азбуки Брайля до сих пор полезно. Чтение (даже по Брайлю) - это приятно, возможность делать заметки разгружает память, а главное, это очень помогает найти работу, помогает найти хоть какую-то занятость. \\
(слайд) В стенах ФабЛаб'а под руководством Глеба Андреевича Мирошника велись разработки тренажёра Брайля - самоучителя для освоения рельефного шрифта в домашних условиях, без преподавателей и дорогого оборудования. Были построены несколько прототипов; непременный элемент тренажёра - ячейка Брайля (слайд, м. б. с видео), которая выводит любой символ шеститочечной системы. Незрячий ощупывает ячейку и учит либо повторяет символы. \\
Глеб Андреевич в своих прототипах использовал пьезоэлектрические ячейки, которые не встречаются в свободной продаже.  (слайд) Наша команда планирует сделать тренажёр с иной ячейкой, где вместо пьезопластинок точки формируются электромоторами. Получается громоздко, но зато такая ячейка Брайля производится из доступных компонентов. Я думаю, родственники незрячего без труда смогут приобрести подобный тренажёр, а, может, и сделать самостоятельно, если у них есть 3D-принтер. В течение школы мы хотим сделать само устройство и написать несколько программ для него,  главным образом - обучающее приложение с использованием распознавания голоса. Я надеюсь, появление такого несложного тренажёра и программ к нему сделают обучение системе Брайля более лёгким делом.\\
Спасибо ФабЛаб'у за предоставление питательной почвы для проекта. Спасибо моим товарищам, пришедшим работать со мной, особенно Ивану и Евгению из Дальневосточного Федерального университета. Евгений будет не только программистом, но и нашим незрячим экспертом. Команда уже сформирована; если Вам очень хочется присоединиться, лучше поговорить со мной отдельно.\\
Благодарю за внимание.
\end{document}