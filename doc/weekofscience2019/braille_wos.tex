%XeLaTeX+makeIndex+BibTeX

\documentclass[a4paper,12pt]{article} %14pt - extarticle
\usepackage[utf8]{inputenc} %русский язык, не менять
\usepackage[T2A, T1]{fontenc} %русский язык, не менять
\usepackage[english, russian]{babel} %русский язык, не менять
\usepackage{fontspec} %различные шрифты
\setmainfont{Times New Roman}
\defaultfontfeatures{Ligatures={TeX},Renderer=Basic}
\usepackage{hyperref} %гиперссылки
\hypersetup{pdfstartview=FitH,  linkcolor=blue, urlcolor=blue, colorlinks=true} %гиперссылки
\usepackage{subfiles}%включение тех-текста
\usepackage{graphicx} %изображения
\usepackage{float}%картинки где угодно
\usepackage{textcomp}
\usepackage{fancyvrb} %fancy verbatim for \VerbatimInput
\usepackage{dsfont}%мат. символы
\newcommand{\ovr}[1]{\overrightarrow{#1}} %своя команда для векторов: \ovr{a} вместо \overrightarrow{w}
\usepackage{listings} %code formatting
\lstset{language=python,
	%keywordstyle=\color{blue},
	%commentstyle=\color{},
	%stringstyle=\color{red},
	tabsize=1,
	breaklines=true,
	columns=fullflexible,
	%numbers=left,
	escapechar=@,
	morekeywords={numpy, np}
}

\begin{document}
	
\section{Устройство вывода тактильной информации для реабилитации незрячих <<Тренажёр Брайля>>}

\textit{Актуальность}.
[Всё больше незрячих, всё меньше грамотных по Брайлю (цифры!)]. 
[Тезисы из аудио из реабилитац. центра].
[Есть дисплеи, но они дороги (ссылки по ГОСТу!). Причины дороговизны: точная электромеханика, мелкосерийное производство].
Предприняты многочисленные попытки создания механических аналогов дисплея Брайля, не получившие развития (ссылки). В данной работе предлагаются конструкции, оптимизированные для изготовления на 3Д-принтере, что упрощает и удешевляет мелкосерийное производство. Исследованы факторы, ограничивающие применение аддитивных технологий в изготовлении миниатюрной механики.

См. прошлые материалы: 
doc/, wiki,
папка на гугл-диске: \url{https://drive.google.com/open?id=1uZ2HyBN039XAgCV-XCHnUukCFEsm3Agg}. 
Особенно см. тезисы к школе ФабЛаб в папке doc/.

\textit{Ход работы}.

a. Первая конструкция. Размеры деталей велики.

б. Вторая конструкция. Размеры уменьшены.

в.

%_________
Дальнейшим развитием механики стал блок вывода брайлевских символов с несколькими ячейками. Расположить ячейки рядом, поместив в корпус по шесть сервоприводов на ячейку, технически сложно. Был спроектирован и изготовлен механизм с подвижной кареткой, которая передвигается вдоль ряда ячеек и переключает каждую по отдельности. Каретка приводится в движение шаговым двигателем типа NEMA17 и несёт три сервопривода SG90, переключающих верхний, средний и нижний ряды точек.

Каждая ячейка полностью состоит из деталей, изготовленных на 3Д\--\-прин\-тере, за исключением штырей, которые изготовлены на заводе из полиэтилена высокой плотности. Общий подвижный узел, вместо отдельных моторов приводящий в действие штыри, позволяет разместить 15-20 ячеек в ряд без усложнения конструкции и существенного повышения стоимости.

Внутри ячейки горизонтальное возвратно-поступательное движение, сообщаемое подвижным блоком, преобразуется в вертикальное (см. рис. ) Размеры символов Брайля увеличены на $50\%$ по сравнению с ГОСТ \cite{gost} по двум причинам: во-первых, изучающим азбуку Брайля (в особенности поздноослепшим) вначале сложно воспринимать мелкий шрифт; во-вторых, попытки создать более миниатюрные подвижные детали не принесли успеха в силу ограничений метода 3Д-печати. Детали, изготовленные на FDM-принтере, имеют слоистую структуру, отчего значительно повышается трение в местах контакта подвижных частей, притом горизонтальные размеры изделий соблюдаются с погрешностью $0.1-0.5$ мм в зависимости от качества сырья. Эти факторы вынуждают оставлять зазоры в $0.4-0.7$ мм между движущимися деталями и не позволяют изготовлять на 3Д-принтере мелкие элементы, например, штыри в ячейках.
%TODO картинки добавить
%TODO картинка с ячейкой (внутреннее устройство)
%TODO распечатать маленькие детали и показать, что дальнейшая миниатюризация с текущим принтером затруднительна
%_________

\textit{Выводы}.
Будем повышать качество 3Д-печати; возможно, пост-обрабатывать. Применим фотополимерную печать. Будем увеличивать скорость работы мультиячеечного дисплея.
%TODO развернуть

\begin{thebibliography}{100}
	
		% https://gsom.spbu.ru/files/upload/library/list_of_literature.pdf - руководство к оформлению списка литературы
	
	\bibitem{gost} ГОСТ Р 56832-2015 Шрифт Брайля. Требования и размеры [Электронный ресурс]. / М.: Стандартинформ, 2016 -. -- Режим доступа:  \url{http://docs.cntd.ru/document/1200129068}, свободный. -Загл. с экрана
\end{thebibliography}

\end{document}