%XeLaTeX+makeIndex+BibTeX

\documentclass[a4paper,12pt]{article} %14pt - extarticle
\usepackage[utf8]{inputenc} %русский язык, не менять
\usepackage[T2A, T1]{fontenc} %русский язык, не менять
\usepackage[english, russian]{babel} %русский язык, не менять
\usepackage{fontspec} %различные шрифты
\setmainfont{Times New Roman}
\defaultfontfeatures{Ligatures={TeX},Renderer=Basic}
\usepackage{hyperref} %гиперссылки
\hypersetup{pdfstartview=FitH,  linkcolor=blue, urlcolor=blue, colorlinks=true} %гиперссылки
\usepackage{subfiles}%включение тех-текста
\usepackage{graphicx} %изображения
\usepackage{float}%картинки где угодно
\usepackage{textcomp}
\usepackage{fancyvrb} %fancy verbatim for \VerbatimInput
\usepackage{dsfont}%мат. символы
\newcommand{\ovr}[1]{\overrightarrow{#1}} %своя команда для векторов: \ovr{a} вместо \overrightarrow{w}
\usepackage{listings} %code formatting
\lstset{language=python,
	%keywordstyle=\color{blue},
	%commentstyle=\color{},
	%stringstyle=\color{red},
	tabsize=1,
	breaklines=true,
	columns=fullflexible,
	%numbers=left,
	escapechar=@,
	morekeywords={numpy, np}
}



\begin{document}
	
\section{Устройство вывода тактильной информации для реабилитации незрячих <<Тренажёр Брайля>>}

\textit{Актуальность}.
[Всё больше незрячих, всё меньше грамотных по Брайлю (цифры!)]. 
[Тезисы из аудио из реабилитац. центра].
[Есть дисплеи, но они дороги (ссылки по ГОСТу!). Причины дороговизны: точная электромеханика, мелкосерийное производство].
Предприняты многочисленные попытки создания механических аналогов дисплея Брайля, не получившие развития (ссылки). В данной работе предлагаются конструкции, оптимизированные для изготовления на 3Д-принтере, что упрощает и удешевляет мелкосерийное производство. Исследованы факторы, ограничивающие применение аддитивных технологий в изготовлении миниатюрной механики.

См. прошлые материалы: doc/, wiki, папка на гугл-диске. Особенно см. тезисы к школе ФабЛаб.

\textit{Ход работы}.

a. Первая конструкция. Размеры деталей велики.

б. Вторая конструкция. Размеры уменьшены.

в. Третья конструкция. Появляются всё более мелкие детали.
Проблемы: непостоянный размер, слоистые (отсюда трение, необходимость зазоров).
Пример: напечатанные штыри разных размеров.

\textit{Выводы}.
Будем повышать качество 3Д-печати; возможно, пост-обрабатывать. Применим фотополимерную печать.

\end{document}