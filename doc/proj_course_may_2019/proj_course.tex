%XeLaTeX+makeIndex+BibTeX OR LuaLaTeX+...
\documentclass[a4paper,12pt]{article} %14pt - extarticle
\usepackage[utf8]{inputenc} %русский язык, не менять
\usepackage[T2A, T1]{fontenc} %русский язык, не менять
\usepackage[english, russian]{babel} %русский язык, не менять
\usepackage{fontspec} %различные шрифты
\setmainfont{Times New Roman}

%\defaultfontfeatures{Ligatures={TeX},Renderer=Basic}
\usepackage[hyphens]{url} %ссылки \url с переносами
\usepackage{hyperref} %гиперссылки href
\hypersetup{pdfstartview=FitH,  linkcolor=blue, urlcolor=blue, colorlinks=true} %гиперссылки
\usepackage{subfiles}%включение тех-текста
\usepackage{graphicx} %изображения
\usepackage{float}%картинки где угодно
\usepackage{textcomp}

\usepackage{dsfont}%мат. символы
%\usepackage[style=gost-footnote]{biblatex} %ГОСТ https://www.ctan.org/pkg/biblatex-gost С ним только хуже! Не использовать его!

\addto\captionsrussian{\def\refname{Использованные источники}}

\begin{document}
\title{Устройство для обучения незрячих чтению рельефным шрифтом <<Тренажёр Брайля>>}
\maketitle
Добрый день. Мы - команда из ИПММ - расскажем о техническом и социальном проекте <<Тренажёр Брайля>>.\\ %титульный слайд
Общественное положение людей с нарушениями зрения даже сейчас нельзя назвать завидным. За примерами далеко ходить не надо. Недалеко от Политеха лет пятнадцать назад ещё работал небольшой завод <<Контакт>>, где были созданы условия для работы слепых. Теперь завод закрыт, здание снесено, на его месте будут возводить жилой дом.\\ %человек с тростью
Вдобавок ко всем несчастьям слепой не может читать и писать. Впрочем, уже давно француз Луи Брайль изобрёл рельефно-точечный шрифт - символы, накалываемые на плотной бумаге (иногда встречаются на упаковках лекарств).\\ %шрифт Брайля
В последние десятилетия, к сожалению, число грамотных по Брайлю падает, хотя слепых в России и мире всё больше. Главная причина - обучение недоступно большинству незрячих. Специализированные школы и центры если и есть, то только в городах, да и в городе слепой по понятным причинам не может ездить далеко от дома. Аппаратура для реабилитации стоит дорого, например, дисплей Брайля - не менее ста тысяч рублей.\\ %слайд с дисплеем Брайля
В мастерской ФабЛаб Политех с 2016 года под руководством Глеба Андреевича Мирошника разрабатывается тренажёр Брайля - аппарат с одной ячейкой, имитирующей любой символ шеститочечного алфавита Брайля. Было создано несколько демонстрационных образцов с пьезоэлектрическими ячейками из фабричного дисплея Брайля.\\ %фото Глеба Андреевича с сайта и видео работы устройства
Зимой мы сделали тренажёр с самодельной механической ячейкой и контроллером Arduino. К нему написали несколько программ на Python и тренажёр мог, общаясь с компьютером через USB, вывести азбуку, часы и несколько уроков с голосовым сопровождением. \\ %фото, м. б. видео работы
В рамках курса мы собрали команду \textit{(Лёша, увы, в Соединённых Штатах)}. Изначально планировали создать два прибора и более развитые программы. Один прибор - улучшенная модель предыдущего, другой - полностью автономный интерактивный прибор под управлением одноплатного компьютера с операционной системой Linux. Особенность тренажёра - низкая стоимость изготовления, 2-4 тысячи рублей за штуку, поэтому спонсоров не искали.\\ %слайд с командой
Начали с создания ячейки. Шесть севоприводов - шесть точек Брайля. Шестерёнки двигают рейки вверх-вниз. \\ %слайд с ячейкой Брайля - с моделью
Затем подключили джойстик, спаяли шестикнопочную клавиатуру, динамик. На лазерном станке вырезали корпус. Сделали накладки на USB-кабель с надписями по Брайлю - "Верх".\\ %Видео и/или фото в проекциях + кабель
Программисты тем временем сделали приложение - заметки, калькулятор, блиц-опрос. Занялись голосовым распознаванием, которое нужно для ответов в проверочных тестах. \\ %просто код и ветки, м. б. ускоренное видео
К апрелю один аппарат был сделан, и мы поехали с ним в Общество Слепых. \\ %фото устройства
В региональной организации Общества Слепых в целом похвалили работу. Но экспертам не понравилась попытка озвучить уроки живым голосом. Сказали, нужен либо профессиональный диктор, либо синтезатор. \\ %видео или фото с Ниной Константиновной (б/звука)
Тогда мы подключили синтезатор RHVoice. Помимо прочего, теперь можно менять скорость воспроизведения.\\ %видео смены скорости
Начали экспериментировать с одноплатным компьютером, но тут не всё пошло по плану. Долго пытались, но не смогли подключить драйвер сервоприводов.\\ %видео
Тогда мы сделали устройство как и раньше, на Arduino, но добавили клавишу пробел и клавишу помощи. Нажав на клавишу помощи, можно услышать подсказки, в каком приложении Вы находитесь и что можно сделать.
Тренажёр имеет размеры 150*150*35 мм, питается от адаптера на 5V 2А. Последний вариант имеет защиту от неправильного подключения: если подключить только провод USB, моторы работать не будут. \\ %слайд со вторым тренажёром или взрыв-схема
В итоге у нас есть достаточно продвинутая механика и развитые программы (более двух тысяч строк кода на GitHub). Но если тренажёр был создан в нескольких экземплярах, всякий раз лучше, то программы строились постепенно с самого начала, и остались некоторые недочёты. Летом я хочу сделать новую модель механики. Осенью планирую вновь собрать команду и переписать программы на Java, в том числе, чтобы можно было подключаться не только к компьютеру, но и к смартфону с Android.\\ % слайды с взрыв-схемой, электрической схемой
Спасибо за внимание. Проект полносьтью открытый, подробную информацию можно найти по ссылке на GitHub. \\ %слайд с благодарностями и ссылкой на GitHub
%Запасные слайды: Умка-01 и Taptilo; Nano Pi; алафвит Брайля; устройство ячейки, старой ячейки, пьезоэлектрической ячейки. Списки библиотек Python, RHVoice. GitHub - ветки, проект. Евгений и нейросеть
\end{document}